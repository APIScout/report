\chapter{Filtering DSL}\label{ch:filtering-dsl}
The filtering DSL compiler is a collection of components that handle the logic for the translation of our DSL into a query that can be understood by the Elasticsearch search API\@.
Let's begin the description of the compiler by understanding the reason behind why we decided to implement it.
This is a tool that is mainly aimed at researchers to perform any kind of empirical study on a subset of our APIs. \\ \\
Let's take a real-world scenario.
Let's say that some researchers are conducting a study on the evolution of security in REST API endpoints over time.
First of all, they want to find some APIs that have more than a couple of commits in their history, as well as a number of paths between 10 and 100.
To do so, they could use the filters in Figure~\ref{fig:filter-commits} to retrieve all APIs with more than two commits in their history, and with a number of paths between 10 and 100.

\begin{figure}[!h]
    \begin{center}
        \verb|api.commits>2 metrics.structure.paths<>(10,100]|
    \end{center}

    \caption{Filter on the number of commits and paths of an API}
    \label{fig:filter-commits}
\end{figure}

\noindent As you can see, the structure of the filter is pretty straight-forward, we have the name of the field to filter for on the left-hand side (\verb|api.commits|), the operator in the middle (\verb|>|), and the value on the right-hand side (\verb|2|).
In the case of the range operator (\verb|<>|), the right-hand side is a little bit more complex.
The round parenthesis indicates that 10 must not be included in the range (thus \verb|>10|), while the square bracket indicates that 100 must be included in the range (thus \verb|<=100|).
Moreover, the space between the filters indicates a logic AND operation.
Thus, the resulting APIs must have more than 2 commits, as well as a number of paths between 10 (not included), and 100 (included). \\ \\
The researchers will now have IDs of several APIs with more than two commits in their history.
What they would like to do now is to retrieve the commit history for those APIs.
To do so, they can do one search request for each API ID in which they use the filter in Figure~\ref{fig:filter-apis}.
This filter tells API Scout to only retrieve those specifications that have that specific API ID -- i.e.\ the commit history for that API\@.

\begin{figure}[!h]
    \begin{center}
        \verb|api.id==306512|
    \end{center}

    \caption{Filter on a specific API ID}
    \label{fig:filter-apis}
\end{figure}

\noindent Finally, the researchers will now be able to analyze how the security of the endpoints evolved in time, since API scout will return them metadata about the specifications, as well as the full JSON specification.
In the following sections, we will describe the inner workings of the language, including the grammar and its components.

\section{Grammar}\label{sec:grammar}
In Figure~\ref{fig:bnf}, we have the BNF (Backus Normal Form) notation of our filtering DSL\@.
It is important to note that part of this grammar -- namely the \textlangle~version \textrangle~and \textlangle~version core \textrangle~non-terminals -- as well as the MAJOR, MINOR, PATCH, PRERELEASE, and BUILD tokens -- are taken from the official semantic versioning page \footnote{https://github.com/semver/semver/blob/master/semver.md}.
As we can see, several tokens are also used to represent terminals in our grammar.
The tokens are SPACE, PARAMETER, STRING, DATE, BOOLEAN, INTEGER, MAJOR, MINOR, PATCH, PRERELEASE, and BUILD\@.
All of these tokens will be discussed more in detail in the following sections of this chapter. \\ \\
In particular, PARAMETER is discussed in Section~\ref{sec:parameters}.
STRING, DATE, BOOLEAN, and INTEGER are discussed in Section~\ref{sec:types}.
MAJOR, MINOR, PATCH, PRERELEASE, and BUILD are discussed in Section~\ref{sec:types}.

\begin{figure}[!h]
    \begin{center}
        \begin{bnf}
            \textlangle~group \textrangle ::= \textlangle~filter \textrangle | \textlangle~filter \textrangle~SPACE \textlangle~group \textrangle
            ;;
            \textlangle~filter \textrangle ::= \textlangle~lhs \textrangle \textlangle~operator \textrangle \textlangle~rhs \textrangle
            ;;
            \textlangle~lhs \textrangle ::= PARAMETER
            ;;
            \textlangle~rhs \textrangle ::= STRING | DATE | BOOLEAN | INTEGER | \textlangle~version \textrangle | \textlangle~range \textrangle
            ;;
            \textlangle~operator \textrangle ::= '==' | '!=' | '\textasciitilde=' | '<>' | '>=' | '<=' | '>' | '<'
            ;;
            \textlangle~range \textrangle ::= \textlangle~bracket \textrangle \textlangle~limit \textrangle~',' \textlangle~limit \textrangle \textlangle~bracket \textrangle
            ;;
            \textlangle~bracket \textrangle ::= '[' | '(' | ']' | ')'
            ;;
            \textlangle~limit \textrangle ::= INTEGER | DATE | \textlangle~version \textrangle
            ;;
            \textlangle~version \textrangle ::= \textlangle~version core \textrangle | \textlangle~version core \textrangle~'-' PRERELEASE | \textlangle~version core \textrangle~'+' BUILD | \textlangle~version core \textrangle~'-' PRERELEASE '+' BUILD
            ;;
            \textlangle~version core \textrangle ::= MAJOR '.' MINOR '.' PATCH
        \end{bnf}
    \end{center}

    \caption{Filtering DSL in the BNF notation}
    \label{fig:bnf}
\end{figure}

\section{Structure}\label{sec:structure}
As we saw in Section~\ref{sec:grammar}, filters are composed of three main parts: a left-hand side -- the parameter, an operator, and a right-hand side -- the value.
To concatenate multiple filters, just separate them with a space.
The space between the filters is interpreted as an AND operator.
The filters described in Listing~\ref{fig:example-filters} are some examples of valid filters.
For example, let's take \verb|api.version.raw<>(1.0.0,2.0.0]|.
In this case, \verb|api.version.raw| is the parameter, \verb|<>| is the range operator, and \verb|(1.0.0,2.0.0]| is the value, in this case the range. \\ \\
In order for the filter to be valid, it must meet several requirements.
These requirements are:

\begin{enumerate}
    \item The parameter must be valid -- i.e.\ it should exist;
    \item The map of supported types of the operator must contain the type of the parameter;
    \item The value's type must match the parameter's type;
    \item In the case of the range operator's right-hand side:
    \begin{itemize}
        \item The range must be surrounded by either square or round brackets, opened to the left of the range and closed to the right of the range;
        \item The range must be composed of only two elements;
        \item The types of the range delimiters must match the ones of the parameter.
    \end{itemize}
\end{enumerate}

\begin{figure}[!h]
    \begin{center}
        \verb|api.version.raw<>(1.0.0,2.0.0]| \\
        \verb|api.name~="weather"| \\
        \verb|api.commits>=10|
    \end{center}

    \caption{Examples of valid filters}
    \label{fig:example-filters}
\end{figure}

\section{Types}\label{sec:types}
The types supported by our service are explained in Table~\ref{tab:types}.
These types are then used to check that the filters are valid.
The types supported by the API Scout DSL range from the most classic types, such as \("\)string\("\), \("\)integer\("\), and \("\)boolean\("\), to more specific ones such as \("\)version\("\), \("\)date\("\), and \("\)range\("\).
Where \("\)version\("\) is taken from the semver website \footnote{https://semver.org/}, and \("\)range\("\) is a custom type made for this DSL\@.

\begin{table}[!h]
    \begin{center}
        \begin{tabular}{l p{15cm}}
            \hline
            \textbf{Name} & \textbf{Description} \\ \hline
            \verb|string| & An alphanumeric set of characters, must be contained inside \verb|""|, and can contain spaces \\
            \verb|integer| & A positive integer number \\
            \verb|boolean| & A \verb|true| or \verb|false| value \\
            \verb|version| & A semantically valid version (must follow the semantic versioning standard~\cite{preston-werner_semantic_nodate}) (e.g.\ 1.0.0-alpha+c3bh3h) \\
            \verb|date| & A date formatted as \verb|dd/mm/yyyy| (e.g.\ 31/12/2021) \\
            \verb|range| & A set of ordered values contained between two limits.
            The limits can be of integer, version, or date type (e.g. \verb|[12,24)|, \verb|[1.0.0,3.0.0]|, or \verb|(15/05/2022,16/07/2023)|) \\ \hline
        \end{tabular}
    \end{center}

    \caption{List of all the supported types}
    \label{tab:types}
\end{table}
\section{Parameters}\label{sec:parameters}
To filter the result, several different categories of parameters are available.
The filter parameters are divided in API, specification, and metrics categories. \\ \\
In the case of the API category, we have all those parameters that are connected to the extracted metadata about the API (not the specification).
In Table~\ref{tab:parameters-api}, we describe all the parameters in this category with their type and a short description.
Table~\ref{tab:parameters-specification} describes the parameters in the specification category.
These are the parameters relative to the metadata extracted from the OpenAPI specification.
Finally, Table~\ref{tab:parameters-metrics} describes the parameters in the metrics category.
These parameters are connected with the metric values of the specifications~\cite{souhaila_serbout_apistic_2024}.

\begin{table}[H]
    \begin{center}
        \begin{tabular}{l l p{9cm}}
            \hline
            \textbf{Name} & \textbf{Type} & \textbf{Description} \\ \hline
            \verb|date| & date & The date when the API was published \\
            \verb|length| & integer, range & The length of the natural language content string of the API  \\
            \verb|api.name| & string & The name of the API \\
            \verb|api.commits| & integer, range & The number of commits found for that API \\
            \verb|api.latest| & boolean & If the given version of the API is the latest \\
            \verb|api.source| & string & The source where the API was extracted from \\
            \verb|api.version.raw| & version, range & The complete version of the api, e.g.\ 1.0.0-alpha.1 \\
            \verb|api.version.valid| & boolean & If the version is compliant with the semantic versioning standard~\cite{preston-werner_semantic_nodate} \\
            \verb|api.version.major| & integer, range & The part of the version containing the major version number \\
            \verb|api.version.minor| & integer, range & The part of the version containing the minor version number \\
            \verb|api.version.patch| & integer, range & The part of the version containing the patch version number \\
            \verb|api.version.prerelease| & string & The part of the version containing the prerelease string \\
            \verb|api.version.build| & string & The part of the version containing the build string \\ \hline
        \end{tabular}
    \end{center}

    \caption{API parameters descriptions}
    \label{tab:parameters-api}
\end{table}

\begin{table}[!h]
    \begin{center}
        \begin{tabular}{l l p{7cm}}
            \hline
            \textbf{Name} & \textbf{Type} & \textbf{Description} \\ \hline
            \verb|specification.type| & string & The type of openAPI specification, can be \verb|specification.openapi| or \verb|swagger| \\
            \verb|specification.version.raw| & version, range & The complete version of the api, e.g.\ 1.0.0-alpha.1 \\
            \verb|specification.version.valid| & boolean & If the version is compliant with the semantic versioning standard~\cite{preston-werner_semantic_nodate} \\
            \verb|specification.version.major| & integer, range & The part of the version containing the major version number \\
            \verb|specification.version.minor| & integer, range & The part of the version containing the minor version number \\
            \verb|specification.version.patch| & integer, range & The part of the version containing the patch version number \\
            \verb|specification.version.prerelease| & string & The part of the version containing the prerelease string \\
            \verb|specification.version.build| & string & The part of the version containing the build string \\ \hline
        \end{tabular}
    \end{center}

    \caption{Specification parameters descriptions}
    \label{tab:parameters-specification}
\end{table}

\begin{table}[!h]
    \begin{center}
        \begin{tabular}{l c p{8cm}}
            \hline
            \textbf{Name} & \textbf{Type} & \textbf{Description} \\ \hline
            \verb|metrics.security.endpoints| & integer, range & The number of API endpoints which explicitly employ specific security schemes~\cite{souhaila_serbout_apistic_2024} \\
            \verb|metrics.schema.models| & integer, range & The number of data models present in the specification \\
            \verb|metrics.schema.properties| & integer, range & The number of properties within the data models in the specification \\
            \verb|metrics.structure.paths| & integer, range & The number of paths in the specification \\
            \verb|metrics.structure.operations| & integer, range & The number of operations in the specification \\
            \verb|metrics.structure.methods| & integer, range & The number of operations in the specification that use HTTP methods (GET, POST, PUT\ldots) \\ \hline
        \end{tabular}
    \end{center}

    \caption{Metrics parameters descriptions}
    \label{tab:parameters-metrics}
\end{table}

\section{Operators}\label{sec:operators}
To complete the DSL structure, we have defined several different operators.
These operators can be used to compare values of specific types with each other.
The operators supported by API Scout are described in Table~\ref{tab:operators}. \\ \\
In the case of the range operation, the parser will decompose the filter into two different filters.
For example, let's take the range \verb|[23,12)|.
Based on the type of parentheses, the parser will create two different filters.
In this case, it will create both a \verb|>=23| and a \verb|<12| filter.
The full list of parentheses conversions is shown in Table~\ref{tab:conversion-table}.

\begin{table}[!h]
    \begin{center}
        \begin{tabular}{c c p{2.5cm} p{9cm}}
            \hline
            \textbf{Operator} & \textbf{Name} & \textbf{Supported Types} & \textbf{Description} \\ \hline
            \verb|==| & Equal & boolean, string, integer, version, date & Check if two values are the same \\
            \verb|!=| & Not Equal & boolean, string, integer, version, date & Check if two values are different \\
            \verb|~=| & Contain & string, version & Check if the right-hand side of the filter is contained in the parameter's value \\
            \verb|<>| & Range & integer, version, date & Check if the right-hand side of the filter is contained in the given range \\
            \verb|>| & Greater & integer, version, date & Check if the right-hand side value is strictly greater than the left-hand side value \\
            \verb|<| & Less & integer, version, date & Check if the right-hand side value is strictly less than the left-hand side value \\
            \verb|>=| & Greater Equal & integer, version, date & Check if the right-hand side value is greater or equal to the left-hand side value \\
            \verb|<=| & Less Equal & integer, version, date & Check if the right-hand side value is less than or equal to the left-hand side value \\ \hline
        \end{tabular}
    \end{center}

    \caption{Operators descriptions}
    \label{tab:operators}
\end{table}

\begin{table}[!h]
    \begin{center}
        \begin{tabular}{c c}
            \hline
            \textbf{Parenthesis} & \textbf{Operator} \\ \hline
            \verb|[| & \verb|>=| \\
            \verb|]| & \verb|<=| \\
            \verb|(| & \verb|>| \\
            \verb|)| & \verb|<| \\ \hline
        \end{tabular}
    \end{center}

    \caption{Parentheses conversion table}
    \label{tab:conversion-table}
\end{table}

