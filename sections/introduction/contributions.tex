\section{Contributions}\label{sec:contributions}
There are several contributions to this project.
In this section, we will identify all of them.

\begin{description}
    \item \textbf{Embedding Method Contribution} This is the choice of embedding method to be used.
    We had to choose an embedding technique that would work in a general case and that would be fast enough to work in an information retrieval system.
    \item \textbf{Filtering Method Contribution} This is the structure and the grammar there is behind the DSL that the tool uses to further refine the user's search query.
    \item \textbf{Engineering Contribution} This is the development of API Scout itself.
    This tool is able to accept either a search query or an OpenAPI specification as input, preprocess it, embed it, and pre-filter the results by using the provided DSL query.
    Finally, it performs an approximate K-NN search in the Elasticsearch index, as well as performing single-document indexing and retrieval.
    \item \textbf{Dataset Contribution} In addition to the MongoDB dataset
    containing the scraped OpenAPI specifications~\cite{souhaila_serbout_apistic_2024}, we have created our own dataset.
    This new dataset is a combination of specification metadata, as well as 512-dimension vectors representing the contents of the documents.
    Moreover, since the MongoDB dataset cannot be modified, we decided to link the original MongoDB documents with the newly created combined documents.
    By doing so, we can have a reference to the complete OpenAPI Specification as well.
    \item \textbf{Validation Contribution} In this final phase of the implementation of API Scout, we performed several experiments to test the performance of our tool.
    In addition to experiments done to test the retrieval performance of the system, we also performed experiments to test the speed of the retrieval of a series of documents.
\end{description}
