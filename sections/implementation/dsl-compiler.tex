\section{DSL Compiler}\label{sec:dsl-compiler}
Since we want to use the filters given us by the users to search the documents in our Elasticsearch database, we need a way of translating our DSL into the Search DSL provided by Elasticsearch.
The conversion of the filters is handled by the Elasticsearch Converter (Figure~\ref{fig:backend-architecture}) component.
To generate the Elasticsearch filters, we have created a mapping between our DSL operators and the operators provided by Elasticsearch (Table~\ref{tab:dsl-to-es}). \\ \\
As stated before, the filter using the range operator will be split up into two different filters, one for the lower bound and one for the upper bound.
For this reason, there is no mapping in Table~\ref{tab:dsl-to-es} for the range operator between the DSL and Elasticsearch. \\ \\
The single filters will then be joined together to form a \verb|bool| query.
The structure of such query is described in Listing~\ref{lst:bool-query}.
The \verb|bool| query contains two sections, \verb|must| and \verb|must_not|.
All the filters with any operator -- except the \verb|!=| operator -- will be added to the \verb|must| array.
The filters with the \verb|!=| operator will be added to the \verb|must_not| array.
In Elasticsearch, the \verb|must| section represents an AND operator, thus a logic AND will be done between all filters contained in that array.
On the other hand, the \verb|must_not| represents a NOT operator. \\ \\
In addition to the user-specified filters, API Scout will -- by default -- add some filters in the query.
In particular, it will add a \verb|api.latest==true| if an \verb|api.commits| filter is not specified, and a \verb|length>200| if a \verb|length| filter is not specified.

\begin{center}
    \begin{lstlisting}[label={lst:bool-query},caption={Structure of an Elasticsearch bool query},captionpos=b,language=json]
                          {
                              "bool": {
                                  "must": [
                                      { ... },
                                      ...
                                  ],
                                  "must_not": [
                                      { ... },
                                      ...
                                  ]
                              }
                          }
    \end{lstlisting}
\end{center}

\begin{table}[!h]
    \begin{center}
        \begin{tabular}{l l}
            \hline
            \textbf{DSL Filter} & \textbf{Elasticsearch Filter} \\ \hline
            \verb|parameter=="value"| & \verb|"term": { "parameter": "value" }| \\
            \verb|parameter!="value"| & \verb|"term": { "parameter": "value" }| \\
            \verb|parameter~="value"| & \verb|"regexp": { "parameter": "value" }| \\
            \verb|parameter>1| & \verb|"range": { "parameter": { "gt": 1 }}| \\
            \verb|parameter<1| & \verb|"range": { "parameter": { "lt": 1 }}| \\
            \verb|parameter>=1| & \verb|"range": { "parameter": { "gte": 1 }}| \\
            \verb|parameter<=1| & \verb|"range": { "parameter": { "lte": 1 }}| \\ \hline
        \end{tabular}
    \end{center}

    \caption{DSL to Elasticsearch conversion table}
    \label{tab:dsl-to-es}
\end{table}

\noindent In API Scout, we can have two different types of search queries, one where a fragment (the natural language query used by K-NN) is present, and one without.
If the fragment is present, then the component will create an Elasticsearch K-NN search query (Listing~\ref{lst:knn-query}), otherwise, it will create a normal Elasticsearch query (Listing~\ref{lst:regular-query}).

\begin{center}
    \begin{lstlisting}[label={lst:knn-query},language=json,caption={Structure of an Elasticsearch K-NN query},captionpos=b]
                    {
                        "from": 1,
                        "size": 10,
                        "knn": {
                            "field": "embedding",
                            "query_vector": [ ... ],
                            "k": 100,
                            "num_candidates": 10000,
                            "filter": {
                                "bool": {
                                    "must": [
                                        { ... },
                                        ...
                                    ],
                                    "must_not": [
                                        { ... },
                                        ...
                                    ]
                                }
                            }
                        }
                    }
    \end{lstlisting}
\end{center}

\begin{center}
    \begin{lstlisting}[label={lst:regular-query},language=json,caption={Structure of a normal Elasticsearch query},captionpos=b]
                        {
                            "from": 1,
                            "size": 10,
                            "query": {
                                "bool": {
                                    "must": [
                                        { ... },
                                        ...
                                    ],
                                    "must_not": [
                                        { ... },
                                        ...
                                    ]
                                }
                            }
                        }
    \end{lstlisting}
\end{center}

