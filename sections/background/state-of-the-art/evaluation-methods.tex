\subsection{Evaluation Methods}\label{subsec:evaluation-methods}
To measure the performance of an information retrieval system, we can use several different metrics.
These metrics can be divided into two distinct categories: offline metrics~\cite{canamares_offline_2020} and online metrics~\cite{hofmann_online_2016}.
Offline metrics are used to evaluate the performance of the system before this is deployed.
On the other hand, online metrics are used to measure performance after the system has been deployed.
Offline evaluation methods can also be divided into two subgroups, order-unaware and order-aware metrics~\cite{canamares_offline_2020}.
To evaluate our system, we used metrics from both subgroups.

\subsubsection{Order-Unaware Metrics}
In order-unaware metrics, the order of the results does not impact the score of the metric.
The most popular order-unaware metrics are precision (Formula ~\ref{eq:precision-redef}), recall (Formula ~\ref{eq:recall-redef}) and F1 score (Formula ~\ref{eq:f1-redef}).

\begin{equation}
    P@K = \frac{\text{Relevant retrieved documents}}{K}
    \label{eq:precision-redef}
\end{equation}

\begin{equation}
    R@K = \frac{\text{Relevant retrieved documents}}{\text{All relevant documents}}
    \label{eq:recall-redef}
\end{equation}

\begin{equation}
    F1@K = 2 \frac{P@K \cdot R@K}{P@K + R@K}
    \label{eq:f1-redef}
\end{equation}

\noindent All three of these metrics have been used in our evaluation process in the variant $@K$.
Where $K$ is the number of retrieved documents.
This was done in order to evaluate the system by increasing $K$ and seeing how the metrics evolved the more documents we retrieved.
Moreover, all of these metrics are computed for a single query.
To get the overall precision of the system, we compute the metric $@K$ for each query and divide it by the total number of queries (Formula~\ref{eq:overall}).

\begin{equation}
    \text{$\overline{Metric}$}@K = \frac{1}{Q}\sum_{q=1}^Q \text{Metric}@K_q
    \label{eq:overall}
\end{equation}

\subsubsection{Order-Aware Metrics}
On the other hand, order-aware metrics take into consideration the ordering of the results when computing the score.
The metrics used by us are the average precision (Formula~\ref{eq:ap}) and the mean average precision (Formula~\ref{eq:map})~\cite{yilmaz_new_2008} -- also in this case in their $@K$ variant.

\begin{equation}
    AP@K = \frac{1}{K} \sum_{k=1}^{K} P@k \cdot rel_k
    \label{eq:ap}
\end{equation}

\noindent In the average precision formula, $K$ is the number of relevant results, and $rel_k$ is the binary relevance score for document $k$.
A score of 0 or 1 is given to all the returned documents.
If a 0 is given, this means that the document is not relevant to the query.
On the other hand, 1 means that the document is relevant.

\begin{equation}
    MAP@K = \frac{1}{Q} \sum_{q=1}^{Q} AP@K_q
    \label{eq:map}
\end{equation}