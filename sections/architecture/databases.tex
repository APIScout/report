\section{Databases}\label{sec:databases}

The system uses two databases to store the data, a MongoDB and an Elasticsearch database.
In the following sections, we will explain more in depth the reasons behind choosing two different databases and how they are used.

\subsection{Specifications Database}\label{subsec:specifications-database}
The specifications database is a MongoDB instance.
Since OpenAPI has multiple versions, and not all API specifications adhere to this standard, it is possible that the specifications have a slightly different structure.
The reason behind choosing only some of the fields of the original dataset is that we consider these to be the most relevant when building a retrieval system and a filtering DSL\@.
We used MongoDB to save the OpenAPI specifications both because MongoDB handles well these kind of JSON/YAML documents, and because other API Analytics services -- outside the scope of this thesis, but which provide the source of the dataset -- rely on such database and data structure~\cite{souhaila_serbout_apistic_2024}. \\ \\
The MongoDB database is composed of two collections, one containing the specifications and one containing the metrics for those specifications.

\subsubsection{Specifications Collection}
In the specifications collection, we have all the specifications that have been scraped from different sources: GitHub, BigQuery, APIs.guru and SwaggerHub~\cite{souhaila_serbout_apistic_2024}.
Moreover, the documents have been saved with two different structures.
The structure of the documents coming from SwaggerHub is described in Listing~\ref{lst:data-swaggerhub} and Table~\ref{tab:data-swaggerhub}.
The structure of the documents coming from GitHub, instead, is described in Listing~\ref{lst:data-github} and Table~\ref{tab:data-github}.
In Tables~\ref{tab:data-swaggerhub} and~\ref{tab:data-github}, we describe what each top-level field in the document means.
In addition, we also highlight those fields that have been used in API Scout.

\begin{lstlisting}[label={lst:data-swaggerhub},language=json,caption={Structure of the data coming from SwaggerHub},captionpos=b]
          {
              "_id": ObjectId("65ce1ae96eb5a9c18ffe38fc"),
              "_fetching_reference": null,
              "_API_reference": "https://api.swaggerhub.com/...",
              "_meta": {},
              "_name": "Bisan Enterprise API",
              "_description": "This is the Bisan...",
              "_created_at": "2020-12-15T09:42:49Z",
              "_last_modified": "2020-12-15T09:42:49Z",
              "_created_by": "mbanna",
              "_API_url": "https://api.swaggerhub.com/.../1.0.1",
              "_version": "1.0.1",
              "_OPENAPI_version": "3.0.0",
              "_API_spec_hash": null,
              "_API_url_hash": "7b88c0cjh4a34vdb78...",
              "api": {}
          }
\end{lstlisting}

\begin{lstlisting}[label={lst:data-github},language=json,caption={Structure of the data coming from GitHub},captionpos=b]
           {
               "_id": ObjectId("65ce1ae96eb5a9c18ffe38fc"),
               "id": 187463,
               "api_spec_id": 472773,
               "api_title": "math-api-v1",
               "api_version": "1.0",
               "commit_date": 2019-11-27T12:31:23.000+00:00,
               "commits": 1,
               "processed_at": 2023-11-27T12:31:23.000+00:00,
               "sha": "vsd876vs68df68fs...",
               "url": "https://raw.githubusercontent.com/...",
               "latest": true,
               "api": {}
           }
\end{lstlisting}

\subsubsection{Metrics Collection}
In the metrics collection, we have one document for each specification.
The connection between the documents is done through the MongoID. If the MongoID in the specification collection and the one in the metrics collection are the same, this means that the two documents are related.
The metrics documents store some statistics about the corresponding OpenAPI specification.
Listing~\ref{lst:data-metrics} and Table~\ref{tab:data-metrics} describe the structure of these documents.
Moreover, as before, we also highlight those fields that have been used in API Scout.

\begin{lstlisting}[label={lst:data-metrics},language=json,caption={Structure of the specification's metrics data},captionpos=b]
            {
                "_id": ObjectId("65ce1ae96eb5a9c18ffe38fc"),
                "securityData": {
                    "schemes": [],
                    "endpointsCount": 20,
                    "average": 10
                },
                "schemas": [],
                "schemaSize": {
                    "schemas": 3,
                    "defined_schemas": 3,
                    "properties": 5,
                    "max_properties": 2,
                    "min_properties": 1
                },
                "structureSize": {
                    "paths": 3,
                    "operations": 3,
                    "used_methods": 3,
                    "used_parameters": 2
                },
                "documentationsData": {}
            }
\end{lstlisting}

\begin{table}[!h]
    \begin{center}
        \begin{tabular}{l p{11cm} c}
            \hline
            \textbf{Field} & \textbf{Description} & \textbf{Used} \\ \hline
            \verb|_id| & The ID given to the document by MongoDB & \cmark \\
            \verb|_fetching_reference| & The ID given by the crawler to the document & \xmark \\
            \verb|_API_reference| & The URL of the API & \xmark \\
            \verb|_meta| & Some metadata on the API server and validity of the JSON specification & \xmark \\
            \verb|_name| & The name of the API (taken from the API specification) & \cmark \\
            \verb|_description| & The description of the API (taken from the API specification) & \xmark \\
            \verb|_created_at| & The date of creation of the API & \cmark \\
            \verb|_last_modified| & The date the API was last modified & \xmark \\
            \verb|_created_by| & The user that uploaded the API specification & \xmark \\
            \verb|_API_url| & The specific URL for the specification version & \xmark \\
            \verb|_version| & The API version & \cmark \\
            \verb|_OPENAPI_version| & The OpenAPI version used in the specification & \cmark \\
            \verb|_API_spec_hash| & The specification hash & \xmark \\
            \verb|_API_url_hash| & The URL hash & \xmark \\
            \verb|api| & The object containing the API specification & \cmark \\ \hline
        \end{tabular}
    \end{center}

    \caption{Description of the data contained in a SwaggerHub document}
    \label{tab:data-swaggerhub}
\end{table}

\begin{table}[!h]
    \begin{center}
        \begin{tabular}{l l c}
            \hline
            \textbf{Field} & \textbf{Description} & \textbf{Used} \\ \hline
            \verb|_id| & The ID given to the document by MongoDB & \cmark \\
            \verb|id| & The ID of the API specification's commit & \xmark \\
            \verb|api_spec_id| & The ID of the API specification & \cmark \\
            \verb|api_title| & The name of the API & \cmark \\
            \verb|api_version| & The version of the API & \cmark \\
            \verb|commit_date| & The time at which this version was committed & \cmark \\
            \verb|commits| & The number of commits this API has in total & \cmark \\
            \verb|processed_at| & The time at which the API was processed by the crawler & \xmark \\
            \verb|sha| & The hash of the specification & \xmark \\
            \verb|url| & The URL of the specification & \xmark \\
            \verb|latest| & If the current commit is the latest or not & \cmark \\
            \verb|api| & The object containing the API specification & \cmark \\ \hline
        \end{tabular}
    \end{center}

    \caption{Description of the data contained in a GitHub document}
    \label{tab:data-github}
\end{table}

\begin{table}[!h]
    \begin{center}
        \begin{tabular}{l p{9.5cm} c}
            \hline
            \textbf{Field} & \textbf{Description} & \textbf{Used} \\ \hline
            \verb|_id| & The ID given to the document by MongoDB & \cmark \\
            \verb|securityData.schemes| & A list of all the security schemes present in the specification & \xmark \\
            \verb|securityData.endpointsCount| & The number of endpoints covered by some kind of security schema & \cmark \\
            \verb|securityData.average| & The average number of endpoints covered by some kind of security schema & \xmark \\
            \verb|schemas| & A list of all the schemas present in the specification & \xmark \\
            \verb|schemaSize.schemas| & The number of schemas used in the specification & \cmark \\
            \verb|schemaSize.defined_schemas| & The number of defined schemas in the specification & \xmark \\
            \verb|schemaSize.properties| & The number of properties used in the specification & \cmark \\
            \verb|schemaSize.max_properties| & The maximum number of properties used in a schema & \xmark \\
            \verb|schemaSize.min_properties| & The minimum number of properties used in a schema & \xmark \\
            \verb|structureSize.paths| & The number of paths in the specification & \cmark \\
            \verb|structureSize.operations| & The number of operations in the specification & \cmark \\
            \verb|structureSize.used_methods| & The number of methods used in the specification & \cmark \\
            \verb|structureSize.used_parameters| & The number of parameters used in the specification & \xmark \\
            \verb|documentationsData| & Object containing metrics about the documentation of endpoints & \xmark \\ \hline
        \end{tabular}
    \end{center}

    \caption{Description of the data contained in a metrics document}
    \label{tab:data-metrics}
\end{table}

\subsection{Embeddings Database}\label{subsec:embeddings-database}
To search for API specifications by querying their natural language descriptions, however, we need to save the vectorization of the natural language extracted from the OpenAPI Specification somewhere.
Since the MongoDB data structure could not be modified, the chosen database for storing the vectors was Elasticsearch -- which also contains metadata and metrics of the API specifications. \\ \\
The reason behind choosing Elasticsearch is that it offers out-of-the-box solutions to deal with textual embeddings.
For instance, the embeddings -- when indexed -- are saved into a Hierarchical Navigable Small World graph (Section ~\ref{subsec:hierarchical-navigable-small-world-graphs}), which helps with retrieval performance.
In addition, it already offers an approximate K-NN search (Section~\ref{subsec:hierarchical-navigable-small-world-graphs}) API with the possibility of pre- or post-filtering the data.
The aforementioned metadata and metrics are used to pre-filter the data in the K-NN search queries.
